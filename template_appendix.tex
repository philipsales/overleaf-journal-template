% Template for Elsevier CRC journal article
% version 1.1-5p dated 18 January 2011
% SEM ACENTUACAO PROBLEMA UTF-8 : NAO ACEITA

% This file (c) 2010-2011 Elsevier Ltd.  Modifications may be freely made,
% provided the edited file is saved under a different name

% This file contains modifications for Procedia Computer Science
% but may easily be adapted to other journals

% Changes since version 1.0
% - elsarticle class option changed from 1p to 3p (to better reflect CRC layout)
% - this version uses option 5p for larger-format journals (text area 24.1 x 18.4 cm)

%-----------------------------------------------------------------------------------

%% This template uses the elsarticle.cls document class and the extension package ecrc.sty
%% For full documentation on usage of elsarticle.cls, consult the documentation "elsdoc.pdf"
%% Further resources available at http://www.elsevier.com/latex

%-----------------------------------------------------------------------------------

%%%%%%%%%%%%%%%%%%%%%%%%%%%%%%%%%%%%%%%%%%%%%%
%%%%%%%%%%%%%%%%%%%%%%%%%%%%%%%%%%%%%%%%%%%%%%
%%                                          %%
%% Important note on usage                  %%
%% -----------------------                  %%
%% This file must be compiled with PDFLaTeX %%
%% Using standard LaTeX will not work!      %%
%%                                          %%
%%%%%%%%%%%%%%%%%%%%%%%%%%%%%%%%%%%%%%%%%%%%%%
%%%%%%%%%%%%%%%%%%%%%%%%%%%%%%%%%%%%%%%%%%%%%%

%% The '5p' and 'times' class options of elsarticle are used for Elsevier CRC
\documentclass[5p,times,authoryear]{elsarticle}

%% The `ecrc' package must be called to make the CRC functionality available
%% ecrc_MIU es el paquete ecrc de Elsevier con modificaciones para la revista RIAI
\usepackage{ecrc_MIU}

%% The ecrc package defines commands needed for running heads and logos.
%% For running heads, you can set the journal name, the volume, the starting page and the authors

%%%%%%%%%%%%%%%%%%%%%%%%%%%%%%%%% Aaadido por Secretaraa RIAI
\usepackage[spanish]{babel}     % Idioma
\addto\captionsspanish{%
\def\tablename{Tabla}%
}
\usepackage[latin1]{inputenc}   % Lengua latina
\usepackage{amsmath}            % Para las referencias a ecuaciones con \eqref
\usepackage{epstopdf}           % Para poder insertar figuras .eps al compilar con PDFLATEX
\usepackage{flushend}           % Para igualar las columnas de la altima pagina
%\usepackage{hyperref}           % Para hipervanculos dentro del PDF
%%%%%%%%%%%%%%%%%%%%%%%%%%%%%%%%%%%%%%%%%%%%%%%%%%%%%%%


%% set the volume if you know. Otherwise `00'
\volume{00}

%% set the starting page if not 1
\firstpage{1}

%% Give the name of the journal
\journalname{Revista Iberoamericana de Automatica e Informática industrial}

%% Give the author list to appear in the running head
%% Example \runauth{C.V. Radhakrishnan et al.}
\runauth{Primer autor et al.}

%% The choice of journal logo is determined by the \jid and \jnltitlelogo commands.
%% A user-supplied logo with the name <\jid>logo.pdf will be inserted if present.
%% e.g. if \jid{yspmi} the system will look for a file yspmilogo.pdf
%% Otherwise the content of \jnltitlelogo will be set between horizontal lines as a default logo

%% Give the abbreviation of the Journal. Contast the Publisher if in doubt what this is.
\jid{RIAI}

%% Give a short journal name for the dummy logo (if needed)
\jnltitlelogo{}

%% Hereafter the template follows `elsarticle'.
%% For more details see the existing template files elsarticle-template-harv.tex and elsarticle-template-num.tex.

%% Elsevier CRC generally uses a numbered reference style
%% For this, the conventions of elsarticle-template-num.tex should be followed (included below)
%% If using BibTeX, use the style file elsarticle-num.bst

%% End of ecrc-specific commands
%%%%%%%%%%%%%%%%%%%%%%%%%%%%%%%%%%%%%%%%%%%%%%%%%%%%%%%%%%%%%%%%%%%%%%%%%%

%% The amssymb package provides various useful mathematical symbols
\usepackage{amssymb}
%% The amsthm package provides extended theorem environments
%% \usepackage{amsthm}

%% The lineno packages adds line numbers. Start line numbering with
%% \begin{linenumbers}, end it with \end{linenumbers}. Or switch it on
%% for the whole article with \linenumbers after \end{frontmatter}.
%% \usepackage{lineno}

%% natbib.sty is loaded by default. However, natbib options can be
%% provided with \biboptions{...} command. Following options are
%% valid:

%%   round  -  round parentheses are used (default)
%%   square -  square brackets are used   [option]
%%   curly  -  curly braces are used      {option}
%%   angle  -  angle brackets are used    <option>
%%   semicolon  -  multiple citations separated by semi-colon
%%   colon  - same as semicolon, an earlier confusion
%%   comma  -  separated by comma
%%   numbers-  selects numerical citations
%%   super  -  numerical citations as superscripts
%%   sort   -  sorts multiple citations according to order in ref. list
%%   sort&compress   -  like sort, but also compresses numerical citations
%%   compress - compresses without sorting
%%
%% \biboptions{comma,round}

% \biboptions{}

% if you have landscape tables
\usepackage[figuresright]{rotating}

% put your own definitions here:
%   \newcommand{\cZ}{\cal{Z}}
%   \newtheorem{def}{Definition}[section]
%   ...

% add words to TeX's hyphenation exception list
%\hyphenation{author another created financial paper re-commend-ed Post-Script}

% para poder introducir varias figuras que ocupen el ancho de las dos columnas.
\usepackage{subfigure}

% declarations for front matter

\begin{document}


%%%%%%%%%%%%%%%%%%%%%%%%%%%%%%%%%%%%%%%%%%%%%%%%%%%%%%%%%%%%%%%%%%%%%%%%%%%%%%%%%%%%%%%%
%% ATTENTION AUTHORS: THIS SECTION IS OBLIGATORY
%%%%%%%%%%%%%%%%%%%%%%%%%%%%%%%%%%%%%%%%%%%%%%%%%%%%%%%%%%%%%%%%%%%%%%%%%%%%%%%%%%%%%%%%

\section*{English Summary}

\textbf{Paper title in English, bold style.}\\

\noindent \textbf{Abstract}\\

Many young learners are required to write essays in English. While most of these students also write essays for other courses in their native language, they often feel hesitant when writing essays in English. This series of four lessons is designed to help students become familiar with writing an essay in English. The first lesson is designed to give students an overview of basic essay writing style. The final three lessons focus on developing skills that are used when analyzing texts as the basis of their essays.\\

\noindent \emph{Keywords:}\\

Keyword 1, keyword 2, keyword 3.
%%%%%%%%%%%%%%%%%%%%%%%%%%%%%%%%%%%%%%%%%%%%%%%%%%%%%%%%%%%%%%%%%%%%%%%%%%%%%%%%%%%%%%%%%%
\section*{Agradecimientos}

Este trabajo ha sido realizado parcialmente gracias al apoyo de la Agencia Nacional (los agradecimientos de financiacian y apoyos han de ser incluidos aqua).

\label{}

%% The Appendices part is started with the command \appendix;
%% appendix sections are then done as normal sections
%% \appendix

%% \section{}
%% \label{}

%% References
%%
%% Following citation commands can be used in the body text:
%% Usage of \cite is as follows:
%%   \cite{key}         ==>>  [#]
%%   \cite[chap. 2]{key} ==>> [#, chap. 2]
%%


%% References with BibTeX database:

\bibliographystyle{elsarticle-harv}
%%\bibliography{bibliography}

%% Authors are advised to use a BibTeX database file for their reference list.
%% The provided style file elsarticle-num.bst formats references in the required Procedia style

%% For references without a BibTeX database:

%%\begin{thebibliography}{01}

%% \bibitem must have the following form:
%%   \bibitem{key}...
%%

% \bibitem{}

%%\bibitem[{Able(1945)}]{Abl:45}
%%Able, B., 1945. Nombre del artículo. Nombre de la revista 35, 123--126.

\begin{thebibliography}{7}
\expandafter\ifx\csname natexlab\endcsname\relax\def\natexlab#1{#1}\fi
\expandafter\ifx\csname url\endcsname\relax
  \def\url#1{\texttt{#1}}\fi
\expandafter\ifx\csname doi\endcsname\relax
  \def\doi#1{\texttt{#1}}\fi
\expandafter\ifx\csname urlprefix\endcsname\relax\def\urlprefix{URL: }\fi
\expandafter\ifx\csname doiprefix\endcsname\relax\def\doiprefix{DOI: }\fi

\bibitem[{Able(1945)}]{Sales:45}
Able, B., 1945. Nombre del artículo. Nombre de la revista 35, 123--126.
\newline\doiprefix\doi{10.3923/ijbc.2010.190.202}

\bibitem[{Able(1956)}]{Sales:56}
Able, B., 1956. Nombre del artículo. Nombre de la revista 135, 7--9.
\newline\doiprefix\doi{10.3923/ijbc.2010.190.202}

\bibitem[{Baker(1963{\natexlab{b}})}]{Bak:63b}
Baker, R., 1963{\natexlab{b}}. Nombre del artículo. Nombre de la revista 34,
  184--186.
\newline\doiprefix\doi{10.3923/ijbc.2010.190.202}

\bibitem[{Charlie y Routh(1966)}]{ChaRou:66}
Charlie, F., Routh, M., 1966. Nombre del artículo. Nombre de la revista 66,
  267--269.
\newline\doiprefix\doi{10.3923/ijbc.2010.190.202}

\bibitem[{García y Martínez(2008)}]{Garcia:2008}
García, F., Martínez, R., 2008. Nombre del artículo. Nombre de la revista
  número, {}números de página.
\newline\doiprefix\doi{10.3923/ijbc.2010.190.202}

\bibitem[{Soukhanov(1992)}]{Heritage:92}
Soukhanov, A.~H. (Ed.), 1992. Nombre de la editorial.


 \end{thebibliography}

\appendix
\section{Primer Apandice}    % Capa Apandice debe tener un tatulo corto.
Este texto esta repetido. Si utiliza Word, use o bien Microsoft Editor de Ecuaciones o
MathType  para las ecuaciones de su artaculo (Insertar | Objeto |
Crear Nuevo | Microsoft Editor de Ecuaciones o Ecuacian MathType).
No debe seleccionar la opcian ``Flotar'' sobre el texto. Por
supuesto, LaTeX gestiona las ecuaciones a travas de macros
pre-programadas.

\section{Segundo Apandice}

Este texto esta repetido. Use el Sistema Internacional como unidades primarias. Se pueden usar
otras unidades como unidades secundarias (entre parantesis). Esto se
aplica a artaculos sobre almacenamiento de datos. Por ejemplo,
escriba ``$15 Gb/cm^2$'' ($100 Gb/in^2$). Se considera una excepcian
cuando las unidades inglesas se usan como identificadores
comerciales, como unidad de disco de 3.5 pulgadas. Evite mezclar
unidades del Sistema Internacional con el Sistema Cegesimal, tales
como corriente en amperios y campo magnatico en  oersteds. Esto a
menudo lleva a confusian porque las ecuaciones no son
dimensionalmente equiparables. Si debe usar unidades mezcladas,
especifique claramente las unidades para cada cantidad  en la
ecuacian.

La unidad en el Sistema Internacional para la fuerza del campo
magnatico H es A/m. Sin embargo, si desea utilizar unidades de $T$,
o bien refiarase a densidad de flujo magnatico $B$ o fuerza del
campo magnatico simbolizado como $\mu_0 H$. Utilice el punto
centrado para separar unidades compuestas,  es decir, $A\cdot m^2$.

\section{Tercer Apandice}

\subsection{Mas sobre Figuras y Tablas}

Este texto esta repetido. Las etiquetas de los ejes de las figuras son a menudo fuentes de
confusian. Utilice palabras en lugar de sambolos. Como ejemplo,
escriba la cantidad ``Magnetizacian,'' o ``Magnetizacian M,'' no
salo ``M.'' Ponga las unidades entre parantesis. No etiquete los
ejes anicamente con unidades. Como en la Fig. 1, por ejemplo,
escriba ``Magnetizacian (A/m)'' o ``Magnetizacian (A $\cdot$
m$^{-1}$),'' no salo ``A/m'' No etiquete los ejes con una relacian
de cantidades y unidades. Por ejemplo, escriba ``Temperatura (K),''
no ``Temperatura/K.''

Los multiplicadores pueden ser especialmente fuente de confusian.
Escriba``Magnetizacian (kA/m)'' o ``Magnetizacian ($10^3$ A/m).'' No
escriba ``Magnetizacian (A/m) $\times$ 1000'' porque el lector no
sabraa si la etiqueta del eje superior en la Fig. 1 es 16000 A/m o
0.016 A/m. Las etiquetas de las figuras deben ser legibles,
aproximadamente de 8 a 12 puntos.

\end{document}

%%
%% End of file `ejemplo latex RIAI.tex'.